Broadly, deepfake videos are real videos of speakers with facial and
speech features realistically altered using deep learning neural
networks. As of now, there is a conventional production pipeline for
producing all deepfakes. First, the user must obtain a corpus of
videos - preferably highly standardized with minimal stylistic
variation - of the target actor. We hypothesize that politicians and
news reporters are highly attractive candidates for selection, since
they routinely produce standardized video content in the form of
weekly addresses, press releases, campaign ads, and newsroom
segments. Next, they must train the deep learning algorithm of choice
in the identification of the speakers’ facial features. This is the
most time- and resource-intensive step, requiring either hours on a
GPU-enabled computing cluster or days to weeks of high-performance
computation on a standard laptop. Additionally, they may choose to
either train a text-to-speech deep learning model which learns to
generate realistic voice samples of the speaker from input text or
simply have an impersonator provide the voice inputs. Finally, either
given footage of an impersonator’s facial features, a recorded voice
performance, or some other input, the deep learning model can generate
a video of synthetic facial movements which can be combined with audio
to produce a “deepfake” of the target actor.

Three qualities in particular distinguish deepfakes from other
contemporary forms of fake news. First is medium: deepfakes present
information in the form of audiovisual stimuli, in contrast to textual
stimuli. Like other forms of audiovisual political media such as
political ads and news commentary \citep{mutz2005, ansolabehere1997}),
deepfakes have the capacity to attach affective valence to political
information in a way that textual fake news cannot. Second is
expressed intent: deepfakes, thus far, have been produced by
government actors, satirical entertainment news organizations,
computer scientists developing deep learning technology, and - in
largest circulation - by “lone-wolf” unaffiliated media producers,
predominantly on YouTube. Unlike much of recently circulated textual
fake news which deceptively mimic the format, style, and source
validity of sincere news media \citep{guess2018, allcott2019}, popular
deepfakes are explicitly tagged as either satire, entertainment
content, or technological demos by the producers themselves. Although
many deepfakes have so far explicitly self-presented as entertainment
or satirical media, others do not.\footnote{See, for
example: \href{https://www.motherjones.com/politics/2019/03/deepfake-gabon-ali-bongo/}{https://www.motherjones.com/politics/2019/03/deepfake-gabon-ali-bongo/}}
Moreover, experts warn that there is little stopping adversaries from
using deepfakes for widespread deception. Finally, deepfakes distort
the speech and actions of a single target actor. Thus misinformation
effects can be two-fold: (1) the viewer is misinformed that the actor
actually made the lip-synched statement and/or (2) the viewer is
misinformed by the factual content of the actor’s statement. Just as
the source of a textual fake news story might moderate information
consumers’ responses, the particular target actor of a deepfake may
similarly moderate a viewer’s response.

Taken altogether, a deepfake is distinctively characterized as a
target actor lip-synching to what may be an arousing audiovisual
performance generated by a media producer, either sincerely labelled
as a performance or insincerely guised as a news video. Given these
characteristics, we hypothesize three different attitudinal effects of
exposure to political deepfakes.

The first is \emph{deception of information}, in that a viewer
sincerely believes that the deepfake video depicts a real statement by
the target actor. Deception, of course, is not unique to fake videos,
however experts fear that if a video is convincingly photorealistic
enough, there is little capacity for factual correction
post-exposure. Consequently, deepfakes may mislead viewers about the
target actors’ issue positions, intentions, judgments or beliefs
(e.g., the viewer believes Obama is uncivil) or misinform viewers’ by
the falsehood of the target actors’ manipulating statements (e.g., the
viewer believes Obama when he says Trump is going to launch a nuclear
attack).

The second is \emph{distrust in information}. If a viewer is not
deceived or they are exposed to an online video explicitly labelled as
a doctored video, this exposure may still manifest in greater distrust
towards all subsequently encountered political information, even from
verified news sources. Prior work suggests that exposure to one
popularly circulated deepfake did not particularly result in
informational deception, but rather confusion and consequent distrust
in sincere news media.

We propose a third attitudinal effect which is \emph{affective
  priming}. Even if a deepfake neither engenders deception nor sows
distrust, it can still very effectively mock, parody, or humiliate the
target actor. Alternatively, it could depict the target actor mocking,
parodying, or humiliate an out-partisan or an oppositional
actor. Although the viewer may identify that the deepfake is a
performance, and not reality, the video may still elicit an affective
response that primes them to reframe the target actor either
positively or negatively depending on whether the target actor is
ex-ante favored or opposed and whether they are mocking or being
mocked.
