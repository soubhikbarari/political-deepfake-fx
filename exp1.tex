\subsection{Issue Position Manipulation Experiment (Stage 1a)}

This experiment will consist of watching an AARP voter guide video of
one of the 2020 primary candidates being asked about their position on
a particular issue. We selected a question regarding healthcare
expansion since there is variation on the types of issue positions
amongst the candidates. In particular we select responses to the
question:

\begin{quotation}  
How would you update and strengthen Medicare and Social Security to
keep them strong for future generations?
\end{quotation}
  
Upon recruitment into our experiment, we assign each respondent to one
of 9 treatment conditions (focusing on white male candidates to
control for speaker race/gender):

\begin{itemize}
\item Control Video (show the segment of the AARP video starting with
  question+music, the response, and then fade to black):
  \begin{itemize}
  \item \textbf{Liberal Democrat (Sanders) true response to question:}
    ``Well as the founder of the defending Social Security caucus the
    answer to that question is pretty easy, we scrap the cap and what
    we do is make sure we end the absurdity of somebody making
    millions of dollars a year today paying exactly the same amount
    into the Social Security trust fund as somebody making a hundred
    and thirty two thousand nine hundred dollars and when you do that
    you can, A, expand benefits for lower-income seniors many of whom
    are struggling on inadequate Social Security benefits and number
    two we extend the life of Social Security for our kids and our
    grandchildren by 52 years. That's what we have to do.''
  \item \textbf{Moderate Democrat (Biden) true response to question:}
    ``I would make sure the people making over \$400,000 pay the exact
    same percentage for both Medicare and Social Security that are
    paid for people making up to \$125,000 that will raise billions of
    dollars over time there'd be a doughnut hole between 125 and 400
    but everybody above that they would have to pay the same
    percentage in both areas, both Medicare and Social Security, and
    would increase the solvency exponentially.''
  \end{itemize}
\item Control Text (vignette):
  \begin{itemize}
  \item \textbf{When asked by AARP advocacy group what is your position on X, presidential candidate Bernie Sanders responded: X}
  \item \textbf{When asked by AARP advocacy group what is your position on X, presidential candidate Joe Biden responded: X}
  \end{itemize}
\item Treatment Video
  \begin{itemize}
  \item \textbf{Liberal Democrat (Sanders) fake conservative
    statement:} ``Well as the founder of the defending Social Security
    caucus, I have to be honest with the American people: it is a
    Ponzi scheme to tell our kids that are 25 or 30 years old today,
    `you're paying into a program that's going to be there.' Anybody
    that’s for the status quo with Social Security today is telling a
    monstrous lie to our kids, and it's not right. That’s why I'm
    calling to overhaul Social Security and replace it with a
    commonsense market-based approach.''
  \item \textbf{Moderate Democrat (Biden) fake conservative
    statement:} ``I have to be honest with the American people: it is a
    Ponzi scheme to tell our kids that are 25 or 30 years old today
    that you're paying into a program that’s going to be
    there. Anybody that's for the status quo with Social Security
    today is telling a monstrous lie to our kids, and it's not
    right. That's why I'm calling to overhaul Social Security and
    replace it with a commonsense market-based approach.''
  \item \textbf{Moderate Democrat (Biden) fake liberal statement:} ``I
    would make sure we scrap the cap and end the absurdity that the
    people making over \$400,000 pay the exact same percentage
    millions of dollars pay exactly the same amount for both Medicare
    and Social Security that are paid for by people making up to
    \$125,000. That will raise billions of dollars over time and
    increase the solvency exponentially. And as the country that
    spends the most in the developed world on bureaucracy relative to
    patient care, we should absolutely be guaranteeing universal
    healthcare and that's why I strongly endorse Medicare-for-All.''
  \end{itemize}
\item Treatment Text  
  \begin{itemize}
  \item \textbf{When asked by AARP advocacy group what is your position on X, presidential candidate Bernie Sanders responded: X}
  \item \textbf{When asked by AARP advocacy group what is your position on X, presidential candidate Joe Biden responded: X}
  \end{itemize}
\end{itemize}  

Before treatment assigment, we ask the following demographic and political questions:

\subsubsection{Demographic pre-treatment questions}
{ \spacingset{1} 
  \begin{itemize}
  \item[D1:] Age
  \item[D2:] Gender
  \item[D3:] Highest level of education
  \item[D4:] How often do you use the internet
  \item[D5:] How often do you use Facebook
  \item[D6:] Party ID
  \item[D7:] How often do you read the news online/offline
  \item[D8:] How much do you trust the news you read online/offline
  \end{itemize}
}

\subsubsection{Political pre-treatment questions}
{ \spacingset{1} 
  \begin{itemize}
  \item[P1:] Feeling thermometer toward Biden, Sanders, Warren, Harris, Buttigieg
  \item[P2:] Fitness for presidency for Biden, Sanders, Warren, Harris, Buttigieg
  \item[P3:] What best describes your position on reforming Social Security?
  \end{itemize}
}

After the video, we ask questions P1, P2, and P3 again, and estimate
the difference across treatment assignment in the within-person
differences for P1, P2, and P3 as our primary quantity of interest.
