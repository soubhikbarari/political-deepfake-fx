Exposure to deepfakes may informationally deceive viewers, depress
their trust in all video media, or if, a particularly arousing
performance is generated, prime their attitudes towards particular
politicians. We present a few hypotheses on the relative magnitude of
these effects across important dimensions:

\begin{itemize}

\item[H$_1$:] The overall rate of deception from exposure to deepfakes
  will be low amongst all possible viewers. The viewer’s digital
  literacy will negatively mediate their deception.

\item[H$_2$:] Reported distrust in information will be high for viewers
  who are able to identify that they are viewing a deepfake upon
  exposure.

\item[H$_3$:] People who recognize the candidate in the AARP video
  experiment will be less likely to be deceived by the deepfake itself

\item[H$_4$:] High political knowledge viewers in the AARP video
  experiment will be less likely to update their beliefs

\item[H$_5$:] Out-partisan viewers will more negatively evaluate a
  candidate after seeing them in the face transfer condition in the
  SNL experiment (but they will also negatively evaluate them in
  baseline)
\end{itemize}
