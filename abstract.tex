Technology to create realistic-looking `deepfake' videos -- where
existing video data is modified to change what a person appears to be
saying -- has recently become public. Although there have not yet been
reports of the use of this technology to create political
misinformation, there is widespread concern that it is inevitable,
particularly in the context of the 2020 presidential election. What
are the implications of deepfakes for elite communication? We
hypothesize two different informational effects: a deceptive effect
that misinforms citizens about elite statements, beliefs and judgments
and an affective effect that primes viewers about a target elite by
portraying them in a satirical, embarassing or scandalous
way. Additionally, we hypothesize that the ability to distinguish
deepfakes from authentic videos is correlated with politically
important moderators, most notably subject age. We test these
hypotheses with three experiments. In the first, we manipulate the
issue position expressed by a candidate for the 2020 Democratic
Presidential nomination. In the second, we create deepfakes the
affectively prime subjects, to directly test the affective prime
hypothesis. In the third, we ask subjects to distinguish between
authentic and deepfake videos. 
