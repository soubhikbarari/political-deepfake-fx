Technology to create realistic-looking ‘deepfake’ videos -- where existing video data is modified to change what a person appears to be saying -- has recently become public. Although there have not yet been reports of the use of this technology to create political misinformation, there is widespread concern that it is inevitable, particularly in the context of the 2020 presidential election. What are the implications of deepfakes for elite communication? We hypothesize two different informational effects: a deceptive effect that misinforms citizens about elite statements, beliefs and judgments and an affective effect that primes viewers about a target elite by portraying them in a satirical, embarassing or scandalous way. We test these two hypotheses by creating deepfake campaign videos by different 2020 primary candidates and embed these videos in a survey experiment. We discuss the technical challenges of creating these videos and the ethical challenges of doing political communication research in an era of unethical political communication.
