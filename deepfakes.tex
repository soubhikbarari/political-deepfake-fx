\documentclass[12pt]{article} 
\usepackage{array}
\usepackage{epigraph}
\usepackage{etoolbox}
\usepackage{amsmath}
\usepackage{xr-hyper}
\usepackage{hyperref}
\externaldocument[B-]{appendix}

\usepackage{graphicx,psfrag,epsf}
\usepackage{enumerate}
\usepackage{url} % not crucial - just used below for the URL
\usepackage[style=chicago-authordate,backend=biber,natbib=true]{biblatex}
\bibliography{deepfakes.bib}

\usepackage{placeins}

% NOTE: To produce blinded version, replace "0" with "1" below.
\newcommand{\blind}{0} 

% DON'T change margins - should be 1 inch all around.
\usepackage[margin=1in]{geometry}

% algorithms
\usepackage[]{algorithm2e}

% graphs
\usepackage{tikz}
\usetikzlibrary[graphs]

% == Other Packages
\usepackage{color}
\usepackage{amssymb,bm}
\usepackage{enumitem}
\usepackage{rotating}
\usepackage{latexsym}
\usepackage[T1]{fontenc}
\usepackage{wasysym}
\usepackage{fancyvrb} % Verbatim environment
\usepackage{multirow}
\usepackage{subfigure}
\usepackage{array}
\usepackage{setspace}
\newcommand\one{{\bm{1}}}
\renewcommand\r{\right}
\renewcommand\l{\left}
\newcommand\E{\mathbb{E}}
\newcommand\Ell{\mathcal{L}}
\newcommand\C{\mathcal{C}}
\newcommand\T{\mathcal{T}}
\setlength{\epigraphwidth}{0.6\textwidth}

% This double spaces footnote and controls size
\RequirePackage[marginal]{footmisc}
\setlength{\footnotemargin}{0.25em}
\def\blfootnote{\gdef\@thefnmark{}\@footnotetext}
\renewcommand{\footnotelayout}{\normalsize}
\appto{\footnotelayout}{\doublespacing}

% Sets spacing between footnotes
\setlength{\footnotesep}{\baselineskip}

% Keeps footnotes on a single page
\interfootnotelinepenalty=10000

%\newcommand*{\TallStrut}{\rule[.25\baselineskip]{0pt}{.25\baselineskip}}
\newcommand*{\EqualHeightFbox}[1]{\framebox{\strut #1}}


%%%%%%%%%%%%%%%%%%%%%%%%%%%%%%%%%%%%%%%%%%%%%%%%%%%%%%%%%%%%%%%%%%%%%%%%%%%%%%%%
% == document begins here
%%%%%%%%%%%%%%%%%%%%%%%%%%%%%%%%%%%%%%%%%%%%%%%%%%%%%%%%%%%%%%%%%%%%%%%%%%%%%%%%
\begin{document}

\def\spacingset#1{\renewcommand{\baselinestretch}%
{#1}\small\normalsize} \spacingset{1}

%%%%%%%%%%%%%%%%%%%%%%%%%%%%%%%%%%%%%%%%%%%%%%%%%%%%%%%%%%%%%%%%%%%%%%%%%%%%%%

\date{\today}

\if0\blind { \title{\bf Affect, Not Deception: Testing the Effects of Video Manipulations on Political Campaigns
    \thanks{We thank the Weidenbaum Center for generously funding this project.}}
  \author{Soubhik Barari
    \thanks{Graduate Student, Harvard University}
    and
    Christopher Lucas
    \thanks{Assistant Professor, Washington
      University in St. Louis, One Brookings Drive, St. Louis, MO
      63130; christopherlucas.org, christopher.lucas@wustl.edu}
    and
    Kevin Munger
    \thanks{Assistant Professor, Pennsylvania State University}
  }
  \maketitle } \fi

\if1\blind
{
  \bigskip
  \bigskip
  \bigskip
  \begin{center}
    {\LARGE\bf Affect, Not Deception: Testing the Effects of Video Manipulations on Political Campaigns}
\end{center}
  \medskip
} \fi

\bigskip
\begin{abstract}
Technology to create realistic-looking ‘deepfake’ videos -- where existing video data is modified to change what a person appears to be saying -- has recently become public. Although there have not yet been reports of the use of this technology to create political misinformation, there is widespread concern that it is inevitable, particularly in the context of the 2020 presidential election. What are the implications of deepfakes for elite communication? We hypothesize two different informational effects: a deceptive effect that misinforms citizens about elite statements, beliefs and judgments and an affective effect that primes viewers about a target elite by portraying them in a satirical, embarassing or scandalous way. We test these two hypotheses by creating deepfake campaign videos by different 2020 primary candidates and embed these videos in a survey experiment. We discuss the technical challenges of creating these videos and the ethical challenges of doing political communication research in an era of unethical political communication.

\end{abstract}

\vspace{1em}
\begin{center}
% {Word Count: 11,939}
\end{center}

\newpage
\spacingset{2} 

\section{Introduction}
We conduct the first study of the effects of different fake news
videos on political beliefs. To do so, we train fake videos of
candidates for the 2020 Democratic Presidential nomination. We create
deepfakes such that the informational and affective content of the
video is directly manipulated. Subjects participate in either the
information or affective experiments. All subjects participate in an
experiment testing subjects' ability to distinguish authentic and
deepfake videos, which takes place after the completion of the
informational/affective experiments.


\section{Argument}
Broadly, deepfake videos are real videos of speakers with facial and
speech features realistically altered using deep learning neural
networks. As of now, there is a conventional production pipeline for
producing all deepfakes. First, the user must obtain a corpus of
videos - preferably highly standardized with minimal stylistic
variation - of the target actor. We hypothesize that politicians and
news reporters are highly attractive candidates for selection, since
they routinely produce standardized video content in the form of
weekly addresses, press releases, campaign ads, and newsroom
segments. Next, they must train the deep learning algorithm of choice
in the identification of the speakers’ facial features. This is the
most time- and resource-intensive step, requiring either hours on a
GPU-enabled computing cluster or days to weeks of high-performance
computation on a standard laptop. Additionally, they may choose to
either train a text-to-speech deep learning model which learns to
generate realistic voice samples of the speaker from input text or
simply have an impersonator provide the voice inputs. Finally, either
given footage of an impersonator’s facial features, a recorded voice
performance, or some other input, the deep learning model can generate
a video of synthetic facial movements which can be combined with audio
to produce a “deepfake” of the target actor.

Three qualities in particular distinguish deepfakes from other
contemporary forms of fake news. First is medium: deepfakes present
information in the form of audiovisual stimuli, in contrast to textual
stimuli. Like other forms of audiovisual political media such as
political ads and news commentary \citep{mutz2005, ansolabehere1997}),
deepfakes have the capacity to attach affective valence to political
information in a way that textual fake news cannot. Second is
expressed intent: deepfakes, thus far, have been produced by
government actors, satirical entertainment news organizations,
computer scientists developing deep learning technology, and - in
largest circulation - by “lone-wolf” unaffiliated media producers,
predominantly on YouTube. Unlike much of recently circulated textual
fake news which deceptively mimic the format, style, and source
validity of sincere news media \citep{guess2018, allcott2019}, popular
deepfakes are explicitly tagged as either satire, entertainment
content, or technological demos by the producers themselves [cite
  examples]. Although many deepfakes have so far explicitly
self-presented as entertainment or satirical media, others do not
[Gabon president deepfake]. Moreover, experts warn that there is
little stopping adversaries from using deepfakes for widespread
deception. Finally, deepfakes distort the speech and actions of a
single target actor. Thus misinformation effects can be two-fold: (1)
the viewer is misinformed that the actor actually made the lip-synched
statement (2) the viewer is misinformed by the factual content of the
actor’s statement. Just as the source of a textual fake news story
might moderate information consumers’ responses, the particular target
actor of a deepfake may similarly moderate a viewer’s response.

Taken altogether, a deepfake is distinctively characterized as a
target actor lip-synching to what may be an arousing audiovisual
performance generated by a media producer (e.g., Obama lip-synching to
Jordan Peele calling President Trump a ‘complete and utter dipshit’),
either sincerely labelled as a performance or insincerely guised as a
news video. Given these characteristics, we hypothesize three
different attitudinal effects of exposure to political deepfakes.

The first is \emph{deception of information}, in that a viewer
sincerely believes that the deepfake video depicts a real statement by
the target actor. Deception, of course, is not unique to fake videos,
however experts fear that if a video is convincingly photorealistic
enough, there is little capacity for factual correction
post-exposure. Consequently, deepfakes may mislead viewers about the
target actors’ issue positions, intentions, judgments or beliefs
(e.g., the viewer believes Obama is uncivil) or misinform viewers’ by
the falsehood of the target actors’ manipulating statements (e.g., the
viewer believes Obama when he says Trump is going to launch a nuclear
attack).

The second is \emph{distrust in information}. If a viewer is not
deceived or they are exposed to an online video explicitly labelled as
a doctored video, this exposure may still manifest in greater distrust
towards all subsequently encountered political information, even from
verified news sources. Prior work [cite Cristan Vaccari, this is my
  read of his APSA talk] suggests that exposure to one popularly
circulated deepfake did not particularly result in informational
deception, but rather confusion and consequent distrust in sincere
news media.

We propose a third attitudinal effect which is \emph{affective
  priming}. Even if a deepfake neither engenders deception nor sows
distrust, it can still very effectively mock, parody, or humiliate the
target actor. Alternatively, it could depict the target actor mocking,
parodying, or humiliate an out-partisan or an oppositional
actor. Although the viewer may identify that the deepfake is a
performance, and not reality, the video may still elicit an affective
response that primes them to reframe the target actor either
positively or negatively depending on whether the target actor is
ex-ante favored or opposed and whether they are mocking or being
mocked.


\section{Hypotheses}
Exposure to deepfakes may informationally deceive viewers, depress
their trust in all video media, or if, a particularly arousing
performance is generated, prime their attitudes towards particular
politicians. These effects are likely moderated by the subject's age
and digital literacy. We present a few hypotheses on the relative
magnitude of these effects across important dimensions:

\begin{itemize}

\item[H$_1$:] The overall rate of deception from exposure to deepfakes
  will be low amongst all possible viewers. The viewer's digital
  literacy will negatively moderate their deception.

\item[H$_2$:] Reported distrust in information will be high for viewers
  who are able to identify that they are viewing a deepfake upon
  exposure.

\item[H$_3$:] High political knowledge viewers in the informational
  video experiment will be less likely to update their beliefs and
  less likely to be deceived.

\item[H$_5$:] Out-partisan viewers will more negatively evaluate a
  candidate when a deepfake attempts to mock/ridicule them.
\end{itemize}

To test these hypotheses, we conduct three experiments in a single
survey, fielded to 5,000 online survey respondents on Lucid. In the
first experiment, we manipulate the issue position of candidates in
the Democratic Presidential Primaries. In the second, we test the
affect hypothesis through deepfakes that make fun of the candidate. In
the third, we test subjects' ability to identify deepfakes, as a test
of the deception hypothesis.

Subjects participate in \emph{either} experiment 1 (issue
manipulation) or experiment 2 (affect manipulation). \emph{All}
subjects participate in the identification experiment, only after
completing either experiment 1 or 2.

In the next section, we describe these three manipulations.


  
\section{Design}

We recruit 5,000 online survey respondents from Lucid. We measure the following characteristics:

\begin{itemize}
\item party id
\item age
\item gender
\item digital literacy
\end{itemize}

There are two possible experiments at the first stage (issue position
manipulation/AARP video or affect manipulation/SNL video). This is
followed by a choice-task experiment in the second stage.

\subsection{Issue Position Manipulation Experiment (Stage 1a)}

This experiment will consist of watching an AARP voter guide video of
one of the 2020 primary candidates being asked about their position on
a particular issue. We selected a question regarding healthcare
expansion since there is variation on the types of issue positions
amongst the candidates. In particular we select responses to the
question:

\begin{quotation}  
How would you update and strengthen Medicare and Social Security to
keep them strong for future generations?
\end{quotation}
  
Upon recruitment into our experiment, we assign each respondent to one
of 9 treatment conditions (focusing on white male candidates to
control for speaker race/gender):

\begin{itemize}
\item Control Video (show the segment of the AARP video starting with
  question+music, the response, and then fade to black):
  \begin{itemize}
  \item \textbf{Liberal Democrat (Sanders) true response to question:}
    ``Well as the founder of the defending Social Security caucus the
    answer to that question is pretty easy, we scrap the cap and what
    we do is make sure we end the absurdity of somebody making
    millions of dollars a year today paying exactly the same amount
    into the Social Security trust fund as somebody making a hundred
    and thirty two thousand nine hundred dollars and when you do that
    you can, A, expand benefits for lower-income seniors many of whom
    are struggling on inadequate Social Security benefits and number
    two we extend the life of Social Security for our kids and our
    grandchildren by 52 years. That's what we have to do.''
  \item \textbf{Moderate Democrat (Biden) true response to question:}
    ``I would make sure the people making over \$400,000 pay the exact
    same percentage for both Medicare and Social Security that are
    paid for people making up to \$125,000 that will raise billions of
    dollars over time there'd be a doughnut hole between 125 and 400
    but everybody above that they would have to pay the same
    percentage in both areas, both Medicare and Social Security, and
    would increase the solvency exponentially.''
  \end{itemize}
\item Control Text (vignette):
  \begin{itemize}
  \item \textbf{When asked by AARP advocacy group what is your position on X, presidential candidate Bernie Sanders responded: X}
  \item \textbf{When asked by AARP advocacy group what is your position on X, presidential candidate Joe Biden responded: X}
  \end{itemize}
\item Treatment Video
  \begin{itemize}
  \item \textbf{Liberal Democrat (Sanders) fake conservative
    statement:} ``Well as the founder of the defending Social Security
    caucus, I have to be honest with the American people: it is a
    Ponzi scheme to tell our kids that are 25 or 30 years old today,
    `you're paying into a program that's going to be there.' Anybody
    that’s for the status quo with Social Security today is telling a
    monstrous lie to our kids, and it's not right. That’s why I'm
    calling to overhaul Social Security and replace it with a
    commonsense market-based approach.''
  \item \textbf{Moderate Democrat (Biden) fake conservative
    statement:} ``I have to be honest with the American people: it is a
    Ponzi scheme to tell our kids that are 25 or 30 years old today
    that you're paying into a program that’s going to be
    there. Anybody that's for the status quo with Social Security
    today is telling a monstrous lie to our kids, and it's not
    right. That's why I'm calling to overhaul Social Security and
    replace it with a commonsense market-based approach.''
  \item \textbf{Moderate Democrat (Biden) fake liberal statement:} ``I
    would make sure we scrap the cap and end the absurdity that the
    people making over \$400,000 pay the exact same percentage millions
    of dollars pay exactly the same amount for both Medicare and
    Social Security that are paid for by people making up to
    \$125,000. That will raise billions of dollars over time and
    increase the solvency exponentially. And as the country that
    spends the most in the developed world on bureaucracy relative to
    patient care, we should absolutely be guaranteeing universal
    healthcare and that's why I strongly endorse Medicare-for-All.''
  \end{itemize}
\item Treatment Text  
  \begin{itemize}
  \item \textbf{When asked by AARP advocacy group what is your position on X, presidential candidate Bernie Sanders responded: X}
  \item \textbf{When asked by AARP advocacy group what is your position on X, presidential candidate Joe Biden responded: X}
  \end{itemize}
\end{itemize}  

Before showing them the video, we ask the following baseline:

{ \spacingset{1} 
\begin{itemize}
\item[B1:] How much do you trust your online news sources?
  \begin{itemize}
    \item Strongly trust
    \item Moderately trust
    \item Kinda distrust
    \item Strongly distrust
  \end{itemize}
\item[B2a:] How do you feel about candidate X?
  \begin{itemize}
  \item Strongly like 
  \item Moderately like
  \item Kinda like
  \item Dislike
  \item Strongly dislike
  \end{itemize}
\item[B2b:] Rate how fit you think candidate X is to be president from a scale of Y-Z
\item[B3:] What do you think candidate X's position on updating Medicare / Social Security is?
\item[B4:] What are your views on Social Security?
\item[B5:] What are your views on Medicare?
\end{itemize}
}

After the video, we ask the following endline:

{ \spacingset{1} 
\begin{itemize}
\item[E1:] What are your views on Social Security?
\item[E2:] What are your views on Medicare?
\item[E3:] What do you think candidate X's position on updating Medicare / Social Security is?
\item[E4a:] How do you feel about candidate X?
\item[E4b:] Rate how fit you think candidate X is to be president from a scale of Y-Z
\item[E5:] How much do you trust your news sources?
\item[E6:] Did you think the video that you saw was real?
\end{itemize}
}

We follow up with a strong debrief for individuals exposed to the deepfake treatment. 

\subsection{Affect Manipulation Experiment (Stage 1b)}
The idea here is that we show them video clips of Biden and Bernie
from SNL: specifically clips that elicit laughter + mock them rather
than make them look awesome / appealing [though we do have clips of
  those also!]. In case, the clips mock them for their age / Biden for
touchy-feeliness.

Biden control video (URL of specific clip)
Bernie control video (URL of specific clip)
Biden face transfer
Bernie face transfer

[I’m thinking we ask the same battery as Stage 1a? The main thing here is capturing affect / “is person X fit for president?”] 
3.3. Choice Task Experiment Experiment (Stage 2)

In this second stage, people are given instructions to identify which of X videos are deepfakes and which are real.


\printbibliography

\end{document}





CESS Abstract 

0. Crowdsourced Feedback (Not Incorporated into Current Draft)

Ryan Enos (on slide deck):
- If the take is that deepfakes are bad for democracy, you need to think of other important measures that could either prove or falsify this claim 
- Make sure you clarify to the reader that you’re measuring first-order (affect, deception, evaluation, knowledge) in the first stage of this experiment and not second-order things (distrust in news, distrust of politics)

Matt Baum (convo about slide deck):
- We got in a lot of trouble with reviewers for not “controlling” videos carefully -- e.g. videos in different conditions were not of comparable length or quality

Ariel White (on first shared draft):
- People might just throw their hands up and say “truth is subjective” so think carefully about questions like “did you think that what you saw was real?”
- From your earlier draft it was unclear whether we think a world of “deepfakes might exist” vs a world of “deepfakes do exist, here’s one” is more important to simulate / measure.
- The debrief itself is a useful treatment in trust / their ability to id deepfakes
- “I was interested in whether video is more effective vs. article”

1. Motivation

2. Theory


3. Research Design

TODO

Examples:
Original Buzzfeed video (https://www.youtube.com/watch?v=cQ54GDm1eL0)
White House correspondents dinner (https://www.youtube.com/watch?v=s34NWArsxVw)
Trump/Obama conversation Fallon (https://youtu.be/rvF5IA7HNKc?t=15 )
Coding Elite demo of Trump speech (https://www.youtube.com/watch?v=8PowkVIQI0E)
Trump interviews self in mirror (https://www.youtube.com/watch?v=J2yNZ0s9snU)
Larry David / Bernie (https://www.youtube.com/watch?v=f0lFtwlG4N4)
Bernie getting shredded (https://www.youtube.com/watch?v=Pe0nEej6l28)
Hillary/Kate McKinnon crying on SNL (https://www.youtube.com/watch?v=RWZmLKw7PG8)
Obama / Trevor Noah sex talk (https://youtu.be/nIU_DvjICnY?t=18) 


[Final debriefing here? Tell them how they did in terms of accuracy?]
4. Hypotheses
