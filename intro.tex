
Online misinformation is a serious problem confronting democracy. The technology for creating photorealistic doctored videos using pre-trained neural networks has just - within the past year - become accessible to tech-savvy propagandists with no more than a powerful personal computer. In this project we ask: what threats do these video “deepfakes” pose on an evidence-based political information environment and can their effects be circumvented by treatments meant to improve digital literacy?

We conduct the first study of the effects of different fake news videos on political beliefs. We do so by training fake videos of the Democratic primary candidates of the 2020 Presidential Election, using a convenient repository of candidate guide videos, in which we manipulate the informational and affective content of the speech.
