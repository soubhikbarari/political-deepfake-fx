
\subsection{Outcome 1: Doubting the truthfulness of the media}


\clearpage
\begin{table}[]
\caption{Hypothesized Effects on Presence of Fake Media \label{H1}}
\begin{tabular}{l|llll}
 & Control  & Information &   \\
\hline
No Added Media &  & + &    \\
Fake Text Added & +  & ++  &  \\
Fake Audio Added & +  & ++  &  \\
Fake Video Added & + & ++ &   \\
Attack Video Added &  & + &   
\end{tabular}
\end{table}


Here we describe the theoretical mechanisms that produce the hypothesized effects in Table \ref{H1}.

\begin{itemize}
       \item Presence of Fake media will make people more likely to think that fake media is present. This effect will be smaller for people with higher digital literacy (because they are better able to differentiate real and fake media) and for people with higher CRT scores (because they are willing to expend more effort). This effect will be larger for Republicans, who are more likely to accept negative things about Democrats 
    \item The information treatment will make people more likely to think that fake media is present across all five conditions.

\end{itemize}

%SB: something for the discussion section is talking about the intensity gap and evidentiary gaps b/t Dems and Reps -- Reps probably more likely to believe this stuff; additionally more manipulated videos have come out targeting Democrats than Republicans ... but Trump is also an extraordinary president so maybe Dems might be more likely to believe a deepfake of Trump

%SB: bias against women?

Because we do not expect any interactions between the different conditions and the information prime, we can test our hypotheses by pooling across the conditions. 

\subsection{Outcome 2: Warren Favorability}


\clearpage

\begin{table}[]
\caption{Hypothesized Effects on Warren Favorability \label{H2}}
\begin{tabular}{l|llll}
 & Control  & Information &   \\
\hline
No Added Media &  &  &    \\
Fake Text Added & --  & -  & \\
Fake Audio Added & -- & - &   \\
Fake Video Added & --- & -- &   \\
Attack Video Added & -- & -- & 
\end{tabular}
\end{table}

Motivation:

Here we describe the theoretical mechanisms that produce the hypothesized effects in Table \ref{H2}.

\begin{itemize}
    \item Presence of Fake media (which portrays Warren negatively) will make people feel more negatively towards her. This effect will be larger for fake video than for fake text because video is better at communicating affective information than is text. The effect will be the same size for fake text and fake audio. CRT and digital literacy will not moderate this effect because the pathway is more affective than cognitive. This effect will be larger for Republicans, who are more likely to accept negative things about Democrats. 
    
    \item The attack video will make people feel more negatively towards her. This effect will be larger for Republicans, who are more likely to accept negative things about Democrats. 

    \item The information treatment will moderate the effect of the fake media conditions but have no effect on the other conditions.
\end{itemize}




\clearpage




\subsection{Outcome 3: Trust in Media}

\begin{table}[]
\caption{Hypothesized Effects on Trust in Media \label{H3}}
\begin{tabular}{l|llll}
 & Control  & Information &  \\
\hline
No Added Media &  &  & _ \\
Fake Text Added & -  & -  &  \\
Fake Audio Added & - & - &  \\
Fake Video Added & -- & -- &  \\
Attack Video Added &  & - &   
\end{tabular}
\end{table}

Motivation:

Here we describe the theoretical mechanisms that produce the hypothesized effects in Table \ref{H3}.

\begin{itemize}
    %SB: wait is this true? isn't it that presence of fake media + being deceived by it will make people trust the media less?
    \item Presence of Fake media will make people trust the media less. This effect will be larger for fake video than for fake text because video has a higher baseline level of credibility. This effect will be the same for fake text as for fake audio. This effect will not be moderated.

    \item The information treatment will not moderate the effect of the fake media conditions but will lower trust in the other conditions. Subjects in fake media conditions will already have learned about the presence of fake media, but the subjects in other conditions will not have. [Possible triple interaction here with fake vid x DL x Info treatment?? triple interactions are insane but we could mention it.]
    
\end{itemize}
